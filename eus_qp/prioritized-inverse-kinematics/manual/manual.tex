%\documentstyle[art10,titlepage,makeidx,twoside,EPSF/epsf,mytabbing]{j-article}

% euslisp
\newif\ifeuslisp
\euslisptrue

%%% added 2004.12.14
\documentclass[]{jarticle}
\usepackage{makeidx,mytabbing,fancyheadings}


\usepackage[dvipdfmx]{graphicx,color,epsfig}
\let\epsfile=\epsfig
\usepackage[dvipdfmx,bookmarks=true,bookmarksnumbered=true,bookmarkstype=toc]{hyperref}
\ifnum 42146=\euc"A4A2 \AtBeginDvi{\special{pdf:tounicode EUC-UCS2}}\else
\AtBeginDvi{\special{pdf:tounicode 90ms-RKSJ-UCS2}}\fi

%%%
\newcommand{\eusversion}{9.00}
\newcommand{\irteusversion}{1.00}


\flushbottom
\makeindex
\pagestyle{myheadings}
\oddsidemargin=0cm
\evensidemargin=0cm

% A4 size
\textwidth=16.5cm
\textheight=24.6cm
\topmargin=-0.8cm
\oddsidemargin= 0.5cm
\evensidemargin=0.5cm

% Letter size
%\topmargin=0.5cm
%\textwidth=17.6cm
%\textheight=23cm
%\oddsidemargin= 0.2cm
%\evensidemargin=0.2cm

\parindent=10pt
\parskip=1mm
%\baselineskip 14pt

\setcounter{totalnumber}{3}
\renewcommand{\topfraction}{0.99}       % 85% of a page (from page
                                        % top)
                                        % can be occupied by tbl / fig
                                        %
\renewcommand{\bottomfraction}{0.99}    % 85% of a page (from page
                                        % bottom)
                                        % can be occupied by tbl / fig
\renewcommand{\textfraction}{0.0}      % text shoud occupy 15% or
                                        % more in
                                        % a page
%\renewcommand{\floatpagefraction}{0.99} % 70% or more shoud occupy in
                                        % a
                                        % a float page


%%% removed 2004.12.14 \jintercharskip=0pt plus3.0pt minus1pt%

\usepackage{amsmath,amssymb}
\usepackage{arydshln}
\usepackage{mathrsfs}
\usepackage{cases} %% subnumcases
\usepackage{enumitem}

\newcommand{\labfig}[1]{\label{fig:#1}}
\newcommand{\labtab}[1]{\label{tab:#1}}
\newcommand{\labeq}[1]{\label{eq:#1}}
\newcommand{\labsec}[1]{\label{sec:#1}}
\newcommand{\labchap}[1]{\label{chap:#1}}
\newcommand{\labitem}[1]{\label{item:#1}}
\newcommand{\figlab}[1]{\labfig{#1}} % alias
\newcommand{\tablab}[1]{\labtab{#1}} % alias
\newcommand{\eqlab}[1]{\labeq{#1}} % alias
\newcommand{\eqlabel}[1]{\labeq{#1}} % alias
\newcommand{\equlab}[1]{\labeq{#1}} % alias
\newcommand{\reffig}[1]{{図~\ref{#1}}~}
\newcommand{\reftab}[1]{{Table~\ref{#1}}~}
\newcommand{\refeq}[1]{{式~(\ref{#1})}~}
\newcommand{\refchap}[1]{第\ref{#1}章}
\newcommand{\refitem}[1]{\ref{#1}}
\newcommand{\refsec}[1]{第\ref{#1}節}
\newcommand{\figref}[1]{\reffig{#1}} % alias
\newcommand{\tabref}[1]{\reftab{#1}} % alias
\renewcommand{\eqref}[1]{\refeq{#1}} % alias
\newcommand{\chapref}[1]{\refchap{#1}} % alias
\newcommand{\secref}[1]{\refsec{#1}} % alias
\newcommand{\bm}[1]{\mbox{\boldmath{$#1$}}}
\makeatletter
\newcommand{\footnoteref}[1]{\protected@xdef\@thefnmark{\ref{#1}}\@footnotemark}
\@addtoreset{equation}{section}
\def\theequation{\thesection.\arabic{equation}}
\makeatother
\newcommand{\eqdef}{\ensuremath{\stackrel{\mathrm{def}}{=}}}
\newcommand{\argmax}{\mathop{\rm arg~max}\limits}
\newcommand{\argmin}{\mathop{\rm arg~min}\limits}


\begin{document}

\newcommand{\ptext}[1]
{\tt \begin{quote} \begin{tabbing} #1 \end{tabbing} \end{quote} \rm}

\newcommand{\desclist}[1]{
\begin{list}{ }{\setlength{\rightmargin}{0mm}\topsep=0mm\partopsep=0mm}
\item #1
\end{list}
\vspace{3mm}}

\newcommand{\functiondescription}[4]{
\index{#1}
{\bf #1} \em #2 \rm \hfill [#3] 
%\if#4 \vspace{3mm} \\ \else \desclist{#4} \fi
%\ifx#4 \vspace{3mm} \\ \else \desclist{#4} \fi
 \desclist{\hspace{0mm}#4}
}

\newcommand{\bfx}[1]{\index{#1}{\bf #1}}
\newcommand{\emx}[1]{\index{#1}{\em #1}}

\newcommand{\longdescription}[3]{
\index{#1}
\begin{emtabbing}
{\bf #1} 
\it #2
\rm
\end{emtabbing}
\desclist{#3}
}

\newcommand{\funcdesc}[3]{\functiondescription{#1}{#2}{function}{#3}}
\newcommand{\macrodesc}[3]{\functiondescription{#1}{#2}{macro}{#3}}
\newcommand{\specialdesc}[3]{\functiondescription{#1}{#2}{special}{#3}}
\newcommand{\methoddesc}[3]{\functiondescription{#1}{#2}{method}{#3}}
\newcommand{\vardesc}[2]{\functiondescription{#1}{}{変数}{#2}}

\newcommand{\fundesc}[2]{\functiondescription{#1}{#2}{function}{\hspace{0mm}}}
\newcommand{\macdesc}[2]{\functiondescription{#1}{#2}{macro}{\hspace{0mm}}}
\newcommand{\spedesc}[2]{\functiondescription{#1}{#2}{special}{\hspace{0mm}}}
\newcommand{\metdesc}[2]{\functiondescription{#1}{#2}{method}{\hspace{0mm}}}

\newcommand{\constdesc}[2]{\functiondescription{#1}{}{定数}{#2}}

\newcommand{\classdesc}[4]{	%class, super slots description
\vspace{2mm} 
\index{#1}
{\Large {\bf #1 }} \hfill [class]  %super
\begin{tabbing}
\hspace{30mm} :super \hspace{5mm} \= {\bf #2} \\
\hspace{30mm} :slots \> #3 
\end{tabbing}
\vspace{4mm}
\desclist{#4}}

\newenvironment{refdesc}{
 \vspace{5mm} \parindent=0mm \topsep=0mm \parskip=0mm \leftmargin=10mm}{
             \parindent=10mm \topsep=3mm \parskip=1mm \leftmargin=0mm }


\date{}
\title{{\LARGE \bf 優先度付き逆運動学による動作生成 \\ リファレンスマニュアル} \\
\vspace{10mm}
{\large \today} \\
}

\author{
平岡直樹 \\
hiraoka@jsk.t.u-tokyo.ac.jp \\
}

\thispagestyle{empty}
\maketitle
\pagenumbering{roman}
\tableofcontents

\newpage
\pagenumbering{arabic}

%%%%%%%%%%%%%%%%%%%%%%
\section{優先度付き逆運動学の基礎} \label{chap:fundamental}
%%%%%
\subsection{優先度付き逆運動学について}

hogehoge

%%%%%
\subsection{章の構成}

\chapref{chap:prioritized-inverse-kinematics}では,クラスを説明する.


%%%%%%%%%%%%%%%%%%%%%%
\section{優先度付き逆運動学} \label{chap:prioritized-inverse-kinematics}
%%%%%
\input{prioritized-inverse-kinematics}

\end{document}

